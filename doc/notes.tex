\documentclass{article}
\usepackage{natbib}
\usepackage{graphicx}
\input{preamble.tex}

\title{Notes on Knockoffs for the Microbiome}
\author{Kris Sankaran}

\begin{document}

There is a tension between modeling and inference in microbiome data analysis.
On the one hand, it has become possible to detect and model complex
relationships between patterns of species abundance and medical characteristics
of interest. However, formal descriptions of the uncertainty associated with
model estimates remains out of reach for all but the simplest (say, generalized
linear or PERMANOVA) models, and typically, to do inference in microbiome
studies, researchers fall back to some type of marginal testing with FDR
control.

The purpose of this note is to describe recent efforts to bridge this gap, based
on the ``knockoffs'' procedure of \citep{}, and to explore the applicability
(and remaining limitations) of recently proposed methods in the microbiome data
analysis context.

Motivation: microbiome species abundance vs. sample characteristics

- has become common to use modeling to study joint abundances -> characteristics
- often also do multiple testing, to quantify uncertainty
- knockoffs are an approach to unify these perspectives

Context
- Originally derived for standard lasso model (relatively recently)
- FDR control idea has been around for a few decades though
- Recently a few different extensions have cropped up: new settings,
assumptions, interpretations


\section{Background}
\label{sec:background}

Standard knockoffs
- setup notation and definitions
- define procedure
- state theorems for FDR control

Model-free knockoffs
- state describe changes in setup
- define changes in procedure (SCIP and Gaussian example)
- state associated theorem

\section{Gaussian model for transformed data}
\label{sec:mf_gaussian_model}

\subsection{Graphical lasso}
\label{subsec:graphical_lasso}


\subsection{High-dimensional factor analysis}
\label{subsec:factor_analysis}

\section{LDA count model}
\label{sec:lda_count_model}



\end{document}
