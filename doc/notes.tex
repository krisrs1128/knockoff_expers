\documentclass{article}
\usepackage{natbib}
\usepackage{graphicx}
\input{preamble.tex}

\title{Notes on Knockoffs for the Microbiome}
\author{Kris Sankaran}

\begin{document}
\maketitle

There is a tension between modeling and inference in microbiome data analysis.
On the one hand, it has become possible to detect and model complex
relationships between patterns of species abundance and medical characteristics
of interest. However, formal descriptions of the uncertainty associated with
model estimates remains out of reach for all but the simplest -- say,
generalized linear or PERMANOVA -- models, and typically, to
do inference in microbiome studies, researchers fall back to some type of
marginal testing with False Discovery Rate (FDR) control \citep{kelly2015power,
  mcmurdie2014waste, benjamini1995controlling, love2014moderated}.

The purpose of this note is to describe recent efforts to bridge this gap, based
on the ``knockoffs'' procedure of \citep{barber2015controlling}, and to explore
the applicability (and remaining limitations) of recently proposed methods in
the microbiome data analysis context. To provide some context,
\citep{barber2015controlling} proposes a procedure to link two ideas that were
both introduced in the mid-1990s -- the FDR-controlling Benjamini-Hochberg
procedure and the sparsity-inducing lasso model \citep{benjamini1995controlling,
  tibshirani1996regression}. More precisely, the knockoffs procedure controls
the FDR of nonzero coefficients estimated by the lasso in a gaussian linear
model. Recently, variants of the knockoffs have been proposed that can be
applied in different settings or to provide different interpretations
\citep{janson2016familywise, candes2016panning, katsevich2017multilayer,
  sesia2017gene, fan2017rank}.

\section{Background}
\label{sec:background}

Standard knockoffs
- setup notation and definitions
- define procedure
- state theorems for FDR control

Model-free knockoffs
- state describe changes in setup
- define changes in procedure (SCIP and Gaussian example)
- state associated theorem

\section{Gaussian model for transformed data}
\label{sec:mf_gaussian_model}

\subsection{Graphical lasso}
\label{subsec:graphical_lasso}


\subsection{High-dimensional factor analysis}
\label{subsec:factor_analysis}

\section{LDA count model}
\label{sec:lda_count_model}

\bibliographystyle{plainnat}
\bibliography{refs.bib}

\end{document}
